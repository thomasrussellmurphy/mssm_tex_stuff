\documentclass[final, letterpaper, 12pt]{article}
\usepackage[utf8]{inputenc}
\usepackage{amsmath}
\usepackage{fullpage}
\usepackage{graphicx}
\usepackage{color}
\author{Hamilton, et al.} 
\date{} 
\title{Hamilton's Helpful Hints} 

%todo: indicates place to do something
%Places we need images:
	%Triangle Geometry
		%Two pole problem
		%basic proportionality
		%altitude to hypotenuse theorem
		%angle bisector theorem
		%special similar triangle
 
\begin{document} 
\maketitle 

\section{Triangle Geometry}	
	 \subsection{Comparing Triangles}\label{sec:using the Pythagorean Theorem}
		Using the Pythagorean Theorem, we can see, for a triangle with sides $a, b, c$: \\
		For a right triangle:
		\begin{equation}
		 	a^2+b^2 = c^2
		\end{equation}
		For an acute triangle:
		\begin{equation}
			a^2+b^2 > c^2
		\end{equation}
		For an obtuse triangle: 
		\begin{equation}
			a^2+b^2 < c^2
		\end{equation}
	 
	 \subsection{Families of Right Triangles}\label{sec: families of right triangles}
		Finding Pythagorean Triples using algebra:
		
		For two integers $m$ and $n$ such that $m > n$ and the two values are relatively prime (they have no common factor except for 1), then the Pythagorean Triple is:
		
		\begin{equation}\label{sec: method of forming Pythagorean Triples}
			(m^2 - n^2, 2mn, m^2 + n^2)
		\end{equation}
		
		Examples of some common triples:
		
		$(3, 4, 5) {\rightarrow} (6, 8, 10) {\rightarrow} (9, 12, 15)$ etc.
		
		$(5, 12, 13)$
		
		$(8, 15, 17)$
		
		$(7, 24, 25)$
		
		$(9, 40, 41)$
		
		$(11, 60, 61)$
		
		$(12, 35, 37)$
		
		$(20, 21, 29)$
		
		$(13, 84, 85)$
	
	\subsection{Special Right Triangles}\label{sec: patterns in special right triangles}
	
		In a $30^{\circ}-60^{\circ}-90^{\circ}$ right triangle, let $s$ be the shorter leg opposing the $30^{\circ}$ angle, $l$ be the longer leg opposing the $60^{\circ}$ angle, and $h$ be the hypotenuse. This means:
		
		\begin{equation}
			s = \frac{1}{2}h
		\end{equation}
		\begin{equation}
			l = s\sqrt{3} = \frac{1}{2}h\sqrt{3}
		\end{equation}
		
		In a $45^{\circ}-45^{\circ}-90^{\circ}$ right triangle, let $s$ be the length of a leg and $h$ be the hypotenuse. This means:
		\begin{equation}
			l\sqrt{2} = h
		\end{equation}
		
	
	\subsection{Altitude to Hypotenuse Theorem}\label{sec: altitude to hypotenuse theorem}
		For a right triangle with legs $a$ and $b$, hypotenuse $c$, the altitude $h$ to side $c$ divides the hypotenuse into two segments $x$ and $y$ that correspond to $b$ and $a$.
		\begin{equation}\label{sec: eqn:similarity relations 1}
			\frac{x}{h} = \frac{h}{y}
		\end{equation}
		\begin{equation}\label{sec: eqn:similarity relations 2}
			\frac{x}{b} = \frac{b}{c}, \frac{y}{a} = \frac{a}{c}
		\end{equation}
		\begin{equation}\label{sec: eqn:similarity relations 3}
			ch = ab 
		\end{equation} 
	
	\subsection{Median to Hypotenuse}\label{sec: median to hypotenuse}
		In a right triangle, when a median is drawn from the vertex containing the right angle to the hypotenuse, the length of the median is one-half the length of the hypotenuse. In addition, the median divides the right triangle into two triangles of equal area.
	
	\subsection{A Special Triangle}\label{sec: special triangle situation}
		In a 13-14-15 triangle, the altitude to the side of length 14 has a length of 12.
	
	\subsection{An Arbitrary Point in an Equilateral Triangle}\label{sec: properties of a point inside an equilateral triangle}
		Let P be any point in the interior of the equilateral triangle. The sum of the distances from P to each side of the triangle is equal to the altitude of the triangle. If $A, B, C$ are the points on each side of the triangle closest to point P and $s$ is the side length of the triangle, then:
		
		\begin{equation}
			PA + PB + PC = h = \frac{1}{2}s\sqrt{3}
		\end{equation}
	
	\subsection{An Important Ratio with an Angle Bisector}\label{sec: a ratio concerning an angle bisector in a triangle}
		In any triangle ABC, where sides $a, b, c$ correspond to the vertices of the same letter, let the angle bisector of C divide side $c$ into two portions, $d, e$, that correspond to sides $a, b$. The relationship between $a, b, d, e$ is:
		\begin{equation}
			\frac{a}{b} = \frac{d}{e}
		\end{equation}
	
	\subsection{The Two-Pole Problem}\label{sec: two poles connected to the base of the other pole}
		If two poles of heights $a, b$ have their bases on a flat surface and the top of b is connected to the bottom of b by a rope and vise versa, the height of pole c from the surface to the intersection of the two pieces of rope is:
		
		\begin{equation}
			c = \frac{ab}{a+b}
		\end{equation}
	
	\subsection{Basic Proportionality Using a Parallel Side}\label{sec: proportionality with a parallel intersection in a triangle}
		Given a triangle CDE with point A on side CE and point B on side DE such that AB is parallel to CD, the following proportions hold true:
		
		\begin{equation}
			\frac{EA}{AC} = \frac{EB}{BD}
		\end{equation}
		\begin{equation}
			\frac{EA}{EC} = \frac{EB}{ED}
		\end{equation}
		\begin{equation}
			\frac{EA}{EC} = \frac{AB}{CD}
		\end{equation}
		\begin{equation}
			\frac{EB}{EC} = \frac{AB}{CD}
		\end{equation}
	
	\subsection{Similar Triangles or Other Shapes}\label{sec: simple proportions between properties of various similar objects}
		For two objects, 1 and 2, let $s$ be a side, $p$ be a perimeter, $h$ be a height, $A$ be an area, and $V$ be a volume. When using some $s_1$ and $s_2$ or other measures of the objects, the measure on one object corresponds to the similar measure on the other. This gives the following formulae:
		
		\begin{equation}
			\frac{s_1}{s_2} = \frac{p_1}{p_2} = \frac{h_1}{h_2}
		\end{equation}
		\begin{equation}
			\left(\frac{s_1}{s_2}\right)^2 = \frac{A_1}{A_2}
		\end{equation}
		\begin{equation}
			\left(\frac{s_!}{s_2}\right)^3 = \frac{V_1}{V_2}
		\end{equation}
	
	\subsection{A Special Triangle Similarity Relationship}\label{sec: a common problem about similar triangles}
		Given a triangle ACD with point B on line AD such that angle ABC has the same measure as ACD, $\Delta ABC ~ \Delta ACD$, so
		
		\begin{equation}
			\frac{AB}{AC} = \frac{AC}{AD}
		\end{equation}
		
\section{Properties of Circles}
	\subsection{Angles In and Around Circles}\label{sec: properties of arcs and angles of tangent and secant lines}
		\input{circlesA1.pdf_tex}
		
		\input{circlesA2.pdf_tex} $x = \frac{1}{2}(a+b)$
		
		\input{circlesA3.pdf_tex} $x = \frac{1}{2}(a-b)$
		
		\input{circlesA4.pdf_tex} $x = \frac{1}{2}(a+b)$
		
		%todo: angle from center, angle with tangent and arc, angle with diameter, angle with chord, two right triangles on circle
		
	\subsection{Tangent and Secant Lines}\label{sec: properties of the lengths of tangent and secant lines}
	
		When a secant line intersects a tangent line outside a circle:
		
		\input{circlesL1.pdf_tex} $x^2 = ab$
		
		When two secant lines intersect outside a circle:
		
		\input{circlesL2.pdf_tex} $c(c+d) = b(a+b)$
		
		When two secant lines intersect inside a circle:
		
		\input{circlesL3.pdf_tex} $ab = cd$
		
	\subsection{Using Radii with Inscribed and Circumscribed Circles}\label{sec: special formulae that involve triangles and circles}
	
		When a circle is inscribed in a triangle:
		
		\input{circlesR1.pdf_tex}
		
		For the semi-perimeter $s = \frac{a+b+c}{2}$, the radius of the circle is:
		\begin{equation}
			r = \sqrt{\frac{(s-a)(s-b)(s-c)}{s}}
		\end{equation}
		
		When a triangle is inscribed in a circle:
		
		\input{circlesR2.pdf_tex}
		
		If the area of the triangle is $K$ (easily found using Heron's Formula), the radius of the circle is:
		\begin{equation}
			r = \frac{abc}{4K}
		\end{equation}
	
	\subsection{Lengths and Areas}\label{sec: arc lengths and the area of segment and sector}
		The circumference of a circle is 
		\begin{equation}
			c = 2 \pi r
		\end{equation}
		
		The length of an arc of angle $\theta$ (in radians) of a circle is
		
		\begin{equation}
			\ell = \theta r
		\end{equation}		
		
\section{Areas of Objects}
	\subsection{Basic Formulae}\label{sec: basic area formulae}
		\begin{equation}
			A_{\Delta} = \frac{1}{2}bh = \frac{1}{2}ab{\text{sin}}(C)
		\end{equation}
		\begin{equation}
			A_\text{rectangle} = bh
		\end{equation}
		\begin{equation}
			A_\text{parallelogram} = \frac{1}{2}d_1d_2 %todo: also a rhombus
		\end{equation}
		\begin{equation}
			A_\text{trapezoid} = \frac{1}{2}h\left(b_1 + b_2\right)
		\end{equation}
		\begin{equation}
			A_\text{equilateral triangle} = \frac{s^2}{4}\sqrt{3}
		\end{equation}
	\subsection{Some More Specific Areas}\label{sec: some interesting area formulae}
		\begin{equation}
			A_\text{polygon} = \frac{1}{2}ap %todo: regular polygon only, explanation
		\end{equation}
		\begin{equation}
			A_\text{circle} = \pi r^2
		\end{equation}
		\begin{equation}
			A_\text{sector} = \frac{arc ^{\circ}}{360^{\circ}}\pi r^2
		\end{equation}
		\begin{equation}
			A_\text{hexagon} = 6A_\text{equilateral triangle} = \frac{3}{2}s^2\sqrt{3}
		\end{equation}
		\begin{equation}
			A_\text{ellipse} = \frac{1}{2}ab{\pi}
		\end{equation}
	\subsection{Special Formulae}\label{sec: Heron's and Brahmagupta's Formulae, Lunes}
	
		For a triangle with sides $a, b, c$ circumscribed by a circle and the semi-perimeter $s = \frac{a+b+c}{2}$, its area is defined by Heron's Formula:
		\begin{equation}
			A = \sqrt{s(s-a)(s-b)(s-c)}
		\end{equation}
		
		For a quadrilateral circumscribed by a circle with sides $a, b, c, d$ and its semi-perimeter $s = \frac{a+b+c+d}{2}$, the area of the quadrilateral is defined by Brahmagupta's Formula:
		\begin{equation}
			A = \sqrt{(s-a)(s-b)(s-c)(s-d)}
		\end{equation}
		
		Two semicircles with diameters equal to each leg of a right triangle are defined to be Lunes.
		\begin{equation}
			A_\text{lunes} = A_{Rt.\Delta}
		\end{equation}
		%todo: 
		
\section{Special Geometric Relationships}
	\subsection{Properties of Polygons}\label{sec: information about angles, diagonals, and other aspects of polygons}
	The degree measure of the interior and exterior angles of an $n$-sided regular polygon are
	
		\begin{equation}
			I = \frac{180^{\circ}(n-2)}{n}
		\end{equation}
		
		\begin{equation}
			E = \frac{360^{\circ}}{n}
		\end{equation}
		
	The total number of diagonals in a convex $n$-gon is
	\begin{equation}
			N_{diagonals} = \frac {n(n-3)}{2}
		\end{equation}
	
	\subsection{The Centers of a Triangle}\label{sec: looking at the centroid, orthocenter, circumcenter, and incenter of a triangle}
		The centroid is the point which is equidistant from each of the vertices of the triangle and is the point at which the three medians of the triangle intersect. It can be found by the mean of the coordinates of the vertices of the triangle. For the points $(x_1, y_1)$, $(x_2, y_2)$, $(x_3, y_3)$, the centroid is the point
		\begin{equation}
			\left(\frac{x_1 + x_2 + x_3}{3}, \frac{y_1 + y_2 + y_3}{3} \right)
		\end{equation}
		
		The distance from a vertex to the centroid is $\frac{2}{3}$ the length of its median.
		In addition, the altitudes of a triangle intersect at the orthocenter, the perpendicular bisectors intersect at the circumcenter, the angle bisectors intersect at the incenter.

	\subsection{Coordinate Geometry}\label{sec: ways of working with coordinates... in geometry}
		To find the distance from a point $P(x_1, y_1)$ to a line $Ax + By + C = 0$ (note the difference from standard form), use the formula:
		
		\begin{equation}
			d = \frac{|Ax_1 + By_1 + C|}{\sqrt{A^2 + B^2}}
		\end{equation}
		
		This formula can be applied to the distance between two parallel lines by choosing any point on one line and finding the distance to the other line.
		
		Given points $A(x_1,y_1)$ and $B(x_2,y_2)$, with point $C$ (the point of division) on segment $\overline{AB}$ and the distance $AC$, when $r = \frac{AC}{AB}$, the coordinates of point $C(x,y)$ are:
		
		\begin{equation}
			\left( x_1 + r (x_2 - x_1), y_1 + r (y_2 - y_1) \right)
		\end{equation}
		
\section{Properties of Three Dimensional Objects}
	\subsection{Prisms and Cylinders}\label{sec: simple formulae for the more simple objects}
		For any right prism or cylinder with with a perimeter of its base $P$, a height $h$, and an area of its bases $B$, its lateral surface area, total surface area, and volume are given by:
		\begin{equation}
			LA = ph
		\end{equation}
		\begin{equation}
			TA = ph+2B
		\end{equation}
		\begin{equation}
			V = Bh
		\end{equation}
	\subsection{Right Cones and Regular Pyramids}\label{sec: looking at the aspects of the pyramids/cones}
	For any right cone or pyramid with with a perimeter of its base $P$, a height $h$, and an area of its base $B$, its lateral surface area, total surface area, and volume are given by:
		\begin{equation}
			LA = \frac{1}{2}p\ell
		\end{equation}
		\begin{equation}
			TA = LA+B
		\end{equation}
		\begin{equation}
			V = \frac{1}{3}Bh
		\end{equation}
	\subsection{Frustums of a Pyramid or Cone}\label{sec: the special formulae relating to frustums}
		In either pyramids or cones, let $P_1, P_2$ be the perimeters of the bases, $B_1, B_2$ be the areas of the bases, $h$ be the height of the frustum, and $\ell$ be the slant height. The lateral surface area, total surface area, and volume of these objects is given by:
		
		\begin{equation}
			LA = \frac{1}{2}{\ell}(P_1 + P_2)
		\end{equation}
		\begin{equation}
			TA = LA + B_1 + B_2
		\end{equation}
		\begin{equation}
			V = \frac{1}{3}h\left(B_1 + B_2 + \sqrt{B_1 B_2}\right)
		\end{equation}
	\subsection{Spheres}\label{sec: properties of a sphere}
	Given the radius $r$ of a sphere, its surface area and volume are
		\begin{equation}
			SA = 4\pi r^2
		\end{equation}
		\begin{equation}
			V = \frac{4}{3}\pi r^3
		\end{equation}
	\subsection{Special Volumes}\label{sec: volumes of regular square pyramid, regular octahedron, regular tetrahedron}
		Let $e$ be the edge length of each regular object.
		
		For a regular square pyramid:
		\begin{equation}
			V = \frac{e^3\sqrt{2}}{6}
		\end{equation}
		
		For a regular octahedron:
		\begin{equation}
			V = \frac{e^3\sqrt{3}}{3}
		\end{equation}
		
		For a regular tetrahedron:
		\begin{equation}
			V = \frac{e^3\sqrt{2}}{12}
		\end{equation}
	\subsection{Special Formulae}\label{sec: euler's formula for polyhedra and solid similarity}
		For a polyhedron with $V$ vertices, $E$ edges, and $F$ faces, the relationship between the three properties is:
		\begin{equation}
			V - E + F = 2
		\end{equation}
		
		For two similar solids with proportional sides $S_1$ and $S_2$ and their respective volumes $V_1$ and $V_2$, the following relationship holds:
		\begin{equation}
			\left( \frac{S_1}{S_2} \right) ^3 = \frac{V_1}{V_2}
		\end{equation}
		
\section{Other Interesting Things}
	\subsection{The Golden Triangle}\label{sec: 36-72-72 triangle and value of phi}
	For an isosceles triangle PQR with $m\angle P = 36 ^{\circ}$ and $\overline{QR} = 1$, the length of $\overline{PR}$ and $\overline{PQ}$ is the golden ratio:
		\begin{equation}
			PR = QR = \phi = \frac{1 + \sqrt{5}}{2}
		\end{equation}
	
	\subsection{Euler's Formula for Polyhedra}\label{sec: a relationship between vertices, edges, and faces in polyhedra}
		Polyhedra are composed of interconnected polygons, creating a system of Vertices ($V$), Edges ($E$), and Faces ($F$) that occur in a precise relationship:
		
		\begin{equation}
			V-E+F = 2
		\end{equation}
	
	\subsection{The Quadratic Formula}\label{sec: one of the most commonly used things in math team}
		Given a quadratic equation in the form $ax^2+bx+c = 0$, the two solutions are:
		\begin{equation}
			x = \frac{-b\pm \sqrt{b^2-4ac}}{2a}
		\end{equation}
	
	\subsection{Ptolemy's Formula for an Inscribed Quadrilateral}\label{sec: formula relating the distances between the vertices of an inscribed quadrilateral}
		For some quadrilateral QPSR inscribed in a circle, the lengths of the segments are related by the formula:
		\begin{equation}
			(PR)(QS) = (PS)(QR) + (PQ)(SR)
		\end{equation}
		
\end{document} 