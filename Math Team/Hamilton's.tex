\documentclass[draft, letterpaper, 12pt]{article}
\usepackage[utf8]{inputenc}
\usepackage{amsmath}
\usepackage{fullpage}
\author{Hamilton, et al.} 
%\date{\today} 
\title{Hamilton's Helpful Hints} 
 
\begin{document} 
\maketitle 
\section{Triangle Geometry}	
	 \subsection{Comparing Triangles}\label{sec:using the Pythagorean Theorem}
		Using the Pythagorean Theorem, we can see, for a triangle with sides $a, b, c$: \\
		For a right triangle:
		\begin{equation}
		 	a^2+b^2 = c^2
		\end{equation}
		For an acute triangle:
		\begin{equation}
			a^2+b^2 > c^2
		\end{equation}
		For an obtuse triangle: 
		\begin{equation}
			a^2+b^2 < c^2
		\end{equation}
	 
	 \subsection{Families of Right Triangles}\label{sec: families of right triangles}
		Finding Pythagorean Triples using algebra:
		
		For two integers $m$ and $n$ such that $m > n$ and the two values are relatively prime (they have no common factor except for 1), then the Pythagorean Triple is:
		
		\begin{equation}\label{sec: method of forming Pythagorean Triples}
			(m^2 - n^2, 2mn, m^2 + n^2)
		\end{equation}
		
		Examples of some common triples:
		
		$(3, 4, 5) {\rightarrow} (6, 8, 10) {\rightarrow} (9, 12, 15)$ etc.
		
		$(5, 12, 13)$
		
		$(8, 15, 17)$
		
		$(7, 24, 25)$
		
		$(9, 40, 41)$
		
		$(11, 60, 61)$
		
		$(12, 35, 37)$
		
		$(20, 21, 29)$
		
		$(13, 84, 85)$
	
	\subsection{Special Right Triangles}\label{sec: patterns in special right triangles}
	
		In a $30^{\circ}-60^{\circ}-90^{\circ}$ right triangle, let $s$ be the shorter leg opposing the $30^{\circ}$ angle, $l$ be the longer leg opposing the $60^{\circ}$ angle, and $h$ be the hypotenuse. This means:
		
		\begin{equation}
			s = \frac{1}{2}h
		\end{equation}
		\begin{equation}
			l = s\sqrt{3} = \frac{1}{2}h\sqrt{3}
		\end{equation}
		
		In a $45^{\circ}-45^{\circ}-90^{\circ}$ right triangle, let $s$ be the length of a leg and $h$ be the hypotenuse. This means:
		\begin{equation}
			l\sqrt{2} = h
		\end{equation}
		
	
	\subsection{Altitude to Hypotenuse Theorem}\label{sec: altitude to hypotenuse theorem}
		For a right triangle with legs $a$ and $b$, hypotenuse $c$, the altitude $h$ to side $c$ divides the hypotenuse into two segments $x$ and $y$ that correspond to $b$ and $a$. \\
		\begin{equation}\label{sec: eqn:similarity relations 1}
			\frac{x}{h} = \frac{h}{y}
		\end{equation}
		\begin{equation}\label{sec: eqn:similarity relations 2}
			\frac{x}{b} = \frac{b}{c}, \frac{y}{a} = \frac{a}{c}
		\end{equation}
		\begin{equation}\label{sec: eqn:similarity relations 3}
			ch = ab 
		\end{equation} 
	
	\subsection{Median to Hypotenuse}\label{sec: median to hypotenuse}
		In a right triangle, when a median is drawn from the vertex containing the right angle to the hypotenuse, the length of the median is one-half the length of the hypotenuse. In addition, the median divides the right triangle into two triangles of equal area.
	
	\subsection{A Special Triangle}\label{sec: special triangle situation}
		In a 13-14-15 triangle, the altitude to the side of length 14 has a length of 12.
	
	\subsection{An Arbitrary Point in an Equilateral Triangle}\label{sec: properties of a point inside an equilateral triangle}
		Let P be any point in the interior of the equilateral triangle. The sum of the distances from P to each side of the triangle is equal to the altitude of the triangle. If $A, B, C$ are the points on each side of the triangle closest to point P and $s$ is the side length of the triangle, then:
		
		\begin{equation}
			PA + PB + PC = h = \frac{1}{2}s\sqrt{3}
		\end{equation}
	
	\subsection{An Important Ratio with an Angle Bisector}\label{sec: a ratio concerning an angle bisector in a triangle}
		In any triangle ABC, where sides $a, b, c$ correspond to the vertices of the same letter, let the angle bisector of C divide side $c$ into two portions, $d, e$, that correspond to sides $a, b$. The relationship between $a, b, d, e$ is:
		\begin{equation}
			\frac{a}{b} = \frac{d}{e}
		\end{equation}
	
	\subsection{The Two-Pole Problem}\label{sec: two poles connected to the base of the other pole}
		If two poles of heights $a, b$ have their bases on a flat surface and the top of b is connected to the bottom of b by a rope and vise versa, the height of pole c from the surface to the intersection of the two pieces of rope is:
		
		\begin{equation}
			c = \frac{ab}{a+b}
		\end{equation}
	
	\subsection{Basic Proportionality Using a Parallel Side}\label{sec: proportionality with a parallel intersection in a triangle}
		Given a triangle CDE with point A on side CE and point B on side DE such that AB is parallel to CD, the following proportions hold true:
		
		\begin{equation}
			\frac{EA}{AC} = \frac{EB}{BD}
		\end{equation}
		\begin{equation}
			\frac{EA}{EC} = \frac{EB}{ED}
		\end{equation}
		\begin{equation}
			\frac{EA}{EC} = \frac{AB}{CD}
		\end{equation}
		\begin{equation}
			\frac{EB}{EC} = \frac{AB}{CD}
		\end{equation}
	
	\subsection{Similar Triangles or Other Shapes}\label{sec: simple proportions between properties of various similar objects}
		For two objects, 1 and 2, let $s$ be a side, $p$ be a perimeter, $h$ be a height, $A$ be an area, and $V$ be a volume. When using some $s_1$ and $s_2$ or other property, the two measures correspond between the two objects. This gives the following formulae:
		
		\begin{equation}
			\frac{s_1}{s_2} = \frac{p_1}{p_2} = \frac{h_1}{h_2}
		\end{equation}
		\begin{equation}
			\left(\frac{s_1}{s_2}\right)^2 = \frac{A_1}{A_2}
		\end{equation}
		\begin{equation}
			\left(\frac{s_!}{s_2}\right)^3 = \frac{V_1}{V_2}
		\end{equation}
	
	\subsection{A Special Similar Triangle}\label{sec: a common problem about similar triangles}
		
	
\end{document} 