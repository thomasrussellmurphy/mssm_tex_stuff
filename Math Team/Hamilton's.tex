\documentclass[draft, letterpaper, 12pt]{article}
\usepackage[utf8]{inputenc}
\usepackage{amsmath}
\usepackage{fullpage}
\usepackage{graphicx}
\author{Hamilton, et al.} 
\date{} 
\title{Hamilton's Helpful Hints} 
 
\begin{document} 
\maketitle 
\section{Triangle Geometry}	
	 \subsection{Comparing Triangles}\label{sec:using the Pythagorean Theorem}
		Using the Pythagorean Theorem, we can see, for a triangle with sides $a, b, c$: \\
		For a right triangle:
		\begin{equation}
		 	a^2+b^2 = c^2
		\end{equation}
		For an acute triangle:
		\begin{equation}
			a^2+b^2 > c^2
		\end{equation}
		For an obtuse triangle: 
		\begin{equation}
			a^2+b^2 < c^2
		\end{equation}
	 
	 \subsection{Families of Right Triangles}\label{sec: families of right triangles}
		Finding Pythagorean Triples using algebra:
		
		For two integers $m$ and $n$ such that $m > n$ and the two values are relatively prime (they have no common factor except for 1), then the Pythagorean Triple is:
		
		\begin{equation}\label{sec: method of forming Pythagorean Triples}
			(m^2 - n^2, 2mn, m^2 + n^2)
		\end{equation}
		
		Examples of some common triples:
		
		$(3, 4, 5) {\rightarrow} (6, 8, 10) {\rightarrow} (9, 12, 15)$ etc.
		
		$(5, 12, 13)$
		
		$(8, 15, 17)$
		
		$(7, 24, 25)$
		
		$(9, 40, 41)$
		
		$(11, 60, 61)$
		
		$(12, 35, 37)$
		
		$(20, 21, 29)$
		
		$(13, 84, 85)$
	
	\subsection{Special Right Triangles}\label{sec: patterns in special right triangles}
	
		In a $30^{\circ}-60^{\circ}-90^{\circ}$ right triangle, let $s$ be the shorter leg opposing the $30^{\circ}$ angle, $l$ be the longer leg opposing the $60^{\circ}$ angle, and $h$ be the hypotenuse. This means:
		
		\begin{equation}
			s = \frac{1}{2}h
		\end{equation}
		\begin{equation}
			l = s\sqrt{3} = \frac{1}{2}h\sqrt{3}
		\end{equation}
		
		In a $45^{\circ}-45^{\circ}-90^{\circ}$ right triangle, let $s$ be the length of a leg and $h$ be the hypotenuse. This means:
		\begin{equation}
			l\sqrt{2} = h
		\end{equation}
		
	
	\subsection{Altitude to Hypotenuse Theorem}\label{sec: altitude to hypotenuse theorem}
		For a right triangle with legs $a$ and $b$, hypotenuse $c$, the altitude $h$ to side $c$ divides the hypotenuse into two segments $x$ and $y$ that correspond to $b$ and $a$.
		\begin{equation}\label{sec: eqn:similarity relations 1}
			\frac{x}{h} = \frac{h}{y}
		\end{equation}
		\begin{equation}\label{sec: eqn:similarity relations 2}
			\frac{x}{b} = \frac{b}{c}, \frac{y}{a} = \frac{a}{c}
		\end{equation}
		\begin{equation}\label{sec: eqn:similarity relations 3}
			ch = ab 
		\end{equation} 
	
	\subsection{Median to Hypotenuse}\label{sec: median to hypotenuse}
		In a right triangle, when a median is drawn from the vertex containing the right angle to the hypotenuse, the length of the median is one-half the length of the hypotenuse. In addition, the median divides the right triangle into two triangles of equal area.
	
	\subsection{A Special Triangle}\label{sec: special triangle situation}
		In a 13-14-15 triangle, the altitude to the side of length 14 has a length of 12.
	
	\subsection{An Arbitrary Point in an Equilateral Triangle}\label{sec: properties of a point inside an equilateral triangle}
		Let P be any point in the interior of the equilateral triangle. The sum of the distances from P to each side of the triangle is equal to the altitude of the triangle. If $A, B, C$ are the points on each side of the triangle closest to point P and $s$ is the side length of the triangle, then:
		
		\begin{equation}
			PA + PB + PC = h = \frac{1}{2}s\sqrt{3}
		\end{equation}
	
	\subsection{An Important Ratio with an Angle Bisector}\label{sec: a ratio concerning an angle bisector in a triangle}
		In any triangle ABC, where sides $a, b, c$ correspond to the vertices of the same letter, let the angle bisector of C divide side $c$ into two portions, $d, e$, that correspond to sides $a, b$. The relationship between $a, b, d, e$ is:
		\begin{equation}
			\frac{a}{b} = \frac{d}{e}
		\end{equation}
	
	\subsection{The Two-Pole Problem}\label{sec: two poles connected to the base of the other pole}
		If two poles of heights $a, b$ have their bases on a flat surface and the top of b is connected to the bottom of b by a rope and vise versa, the height of pole c from the surface to the intersection of the two pieces of rope is:
		
		\begin{equation}
			c = \frac{ab}{a+b}
		\end{equation}
	
	\subsection{Basic Proportionality Using a Parallel Side}\label{sec: proportionality with a parallel intersection in a triangle}
		Given a triangle CDE with point A on side CE and point B on side DE such that AB is parallel to CD, the following proportions hold true:
		
		\begin{equation}
			\frac{EA}{AC} = \frac{EB}{BD}
		\end{equation}
		\begin{equation}
			\frac{EA}{EC} = \frac{EB}{ED}
		\end{equation}
		\begin{equation}
			\frac{EA}{EC} = \frac{AB}{CD}
		\end{equation}
		\begin{equation}
			\frac{EB}{EC} = \frac{AB}{CD}
		\end{equation}
	
	\subsection{Similar Triangles or Other Shapes}\label{sec: simple proportions between properties of various similar objects}
		For two objects, 1 and 2, let $s$ be a side, $p$ be a perimeter, $h$ be a height, $A$ be an area, and $V$ be a volume. When using some $s_1$ and $s_2$ or other measures of the objects, the measure on one object corresponds to the similar measure on the other. This gives the following formulae:
		
		\begin{equation}
			\frac{s_1}{s_2} = \frac{p_1}{p_2} = \frac{h_1}{h_2}
		\end{equation}
		\begin{equation}
			\left(\frac{s_1}{s_2}\right)^2 = \frac{A_1}{A_2}
		\end{equation}
		\begin{equation}
			\left(\frac{s_!}{s_2}\right)^3 = \frac{V_1}{V_2}
		\end{equation}
	
	\subsection{A Special Triangle Similarity Relationship}\label{sec: a common problem about similar triangles}
		Given a triangle ACD with point B on line AD such that angle ABC has the same measure as ACD, $\Delta ABC ~ \Delta ACD$, so
		
		\begin{equation}
			\frac{AB}{AC} = \frac{AC}{AD}
		\end{equation}
\section{Properties of Circles}
	\subsection{Angles In and Around Circles}\label{sec: properties of arcs and angles of tangent and secant lines}
	\subsection{Tangent and Secant Lines}\label{sec: properties of the lengths of tangent and secant lines}
	\subsection{Using Radii with Inscribed and Circumscribed Circles}\label{sec: special formulae that involve triangles and circles}
\section{Areas of Objects}
	\subsection{Basic Formulae}\label{sec: basic area formulae}
		\begin{equation}
			A_{\delta} = \frac{1}{2}bh
		\end{equation}
		\begin{equation}
			A_{rectangle} = bh
		\end{equation}
		\begin{equation}
			A_{parallelogram} = \frac{1}{2}d_1d_2
		\end{equation}
		\begin{equation}
			A_{trapezoid} = \frac{1}{2}h\left(b_1 + b_2\right)
		\end{equation}
		\begin{equation}
			A_{equilateral triangle} = \frac{s^2}{4}\sqrt{3}
		\end{equation}
	\subsection{Some More Specific Areas}\label{sec: some interesting area formulae}
		\begin{equation}
			A_{polygon} = \frac{1}{2}ap
		\end{equation}
		\begin{equation}
			A_{circle} = \pi r^2
		\end{equation}
		\begin{equation}
			A_{sector} = \frac{arc ^{\circ}}{360^{\circ}}\pi r^2
		\end{equation}
		\begin{equation}
			A_{hexagon} = 6A_{equilateral triangle} = \frac{3}{2}s^2\sqrt{3}
		\end{equation}
	\subsection{Special Formulae}\label{sec: Heron's and Brahmagupta's Formulae, Lunes}
	
		For a triangle circumscribed by a circle with sides $a, b, c$ and its semi-perimeter $s = \frac{a+b+c}{2}$, the area of the triangle is:
		\begin{equation}
			A = \sqrt{s(s-a)(s-b)(s-c)}
		\end{equation}
		
		For a quadrilateral circumscribed by a circle with sides $a, b, c, d$ and its semi-perimeter $s = \frac{a+b+c+d}{2}$, the area of the quadrilateral is:
		\begin{equation}
			A = \sqrt{(s-a)(s-b)(s-c)(s-d)}
		\end{equation}
		
		Two semicircles with diameters equal to each leg of a right triangle are defined to be Lunes.
		\begin{equation}
			A_{lunes} = A_{Rt.\Delta}
		\end{equation}
\section{Special Geometric Relationships}
	\subsection{Properties of Polygons}\label{sec: information about angles, diagonals, and other aspects of polygons}
	\subsection{The Centers of a Triangle}\label{sec: looking at the centroid, orthocenter, circumcenter, and incenter of a triangle}
	\subsection{Coordinate Geometry}\label{sec: ways of working with coordinates... in geometry}
\section{Properties of Three Dimensional Objects}
	\subsection{Prisms and Cylinders}\label{sec: simple formulae for the more simple objects}
		For any right prism or cylinder with with a perimeter of its base $P$, a height $h$, and an area of its bases $B$, its lateral surface area, total surface area, and volume are given by:
		\begin{equation}
			LA = ph
		\end{equation}
		\begin{equation}
			TA = ph+2B
		\end{equation}
		\begin{equation}
			V = Bh
		\end{equation}
	\subsection{Right Cones and Regular Pyramids}\label{sec: looking at the aspects of the pyramids/cones}
		\begin{equation}
			LA = \frac{1}{2}p\ell
		\end{equation}
		\begin{equation}
			TA = LA+B
		\end{equation}
		\begin{equation}
			V = \frac{1}{3}Bh
		\end{equation}
	\subsection{Frustums of a Pyramid or Cone}\label{sec: the special formulae relating to frustums}
		In either pyramids or cones, let $P_1, P_2$ be the perimeters of the bases, $B_1, B_2$ be the areas of the bases, $h$ be the height of the frustum, and $\ell$ be the slant height. The lateral surface area, total surface area, and volume of these objects is given by:
		
		\begin{equation}
			LA = \frac{1}{2}{\ell}(P_1 + P_2)
		\end{equation}
		\begin{equation}
			TA = LA + B_1 + B_2
		\end{equation}
		\begin{equation}
			V = \frac{1}{3}h\left(B_1 + B_2 + \sqrt{B_1 B_2}\right)
		\end{equation}
	\subsection{Spheres}\label{sec: properties of a sphere}
		\begin{equation}
			SA = 4\pi r^2
		\end{equation}
		\begin{equation}
			V = \frac{4}{3}\pi r^3
		\end{equation}
	\subsection{Special Volumes}\label{sec: volumes of regular square pyramid, regular octahedron, regular tetrahedron}
		Let $e$ be the edge length of each regular object.
		
		For a regular square pyramid:
		\begin{equation}
			V = \frac{e^3\sqrt{2}}{6}
		\end{equation}
		
		For a regular octahedron:
		\begin{equation}
			V = \frac{e^3\sqrt{3}}{3}
		\end{equation}
		
		For a regular tetrahedron:
		\begin{equation}
			V = \frac{e^3\sqrt{2}}{12}
		\end{equation}
\section{Other Interesting Things}
	\subsection{The Golden Triangle}\label{sec: 36-72-72 triangle and value of phi}
	\subsection{Euler's Formula for Polyhedra}\label{sec: a relationship between vertices, edges, and faces in polyhedra}
		Polyhedra are composed of interconnected polygons, creating a system of Vertices ($V$), Edges ($E$), and Faces ($F$) that occur in a precise relationship:
		
		\begin{equation}
			V-E+F = 2
		\end{equation}
	\subsection{The Quadratic Formula}\label{sec: one of the most commonly used things in math team}
		Given a quadratic equation in the form $ax^2+bx+c = 0$, the two solutions are:
		\begin{equation}
			x = \frac{-b\pm \sqrt{b^2-4ac}}{2a}
		\end{equation}
	\subsection{Ptolemy's Formula for an Inscribed Quadrilateral}\label{sec: formula relating the distances between the vertices of an inscribed quadrilateral}
		For some quadrilateral QPSR inscribed in a circle, the lengths of the segments are related by the formula:
		\begin{equation}
			(PR)(QS) = (PS)(QR) + (PQ)(SR)
		\end{equation}
\end{document} 